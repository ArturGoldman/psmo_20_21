\documentclass[10pt, a4paper]{extarticle}

%% Язык
\usepackage{cmap} % Поиск в PDF
\usepackage{mathtext} % Кириллица в формулах
\usepackage[T2A]{fontenc} % Кодировка
\usepackage[utf8]{inputenc} % Кодировка
\usepackage[english,russian]{babel} % Локализация, переносы

%% Шрифты

% Serif
%\usepackage{euscript} % Шрифт Евклид
%\usepackage{mathrsfs} % Шрифт для математики
\usepackage{libertinus}

% Sans-serif
%\renewcommand{\rmdefault}{cmss}
%\renewcommand{\ttdefault}{cmss}
%\usepackage{sfmath}

% Настройки для xelatex
%\usepackage{polyglossia} % Для выбора языка в xelatex
%\setmainlanguage{russian}
%\setotherlanguages{english}
% Ligatures=TeX is on by default
% https://tex.stackexchange.com/questions/323542/
%\setmainfont[Ligatures=TeX]{Cantarell}
%\newfontfamily{\cyrillicfonttt}{Times New Roman}
%\newfontfamily\cyrillicfont{Cantarell}[Script=Cyrillic]
%\setsansfont[Ligatures=TeX]{Cantarell}
%\newfontfamily\cyrillicfontsf{Cantarell}[Script=Cyrillic]
%\setmonofont{Courier New}
%\newfontfamily\cyrillicfonttt{Courier New}[Script=Cyrillic]

%% Математика
\usepackage{amsmath, amsfonts, amssymb, amsthm, mathtools}
\usepackage{icomma}

% Операторы
\DeclareMathOperator*\plim{plim}
\DeclareMathOperator{\sgn}{sign}
\DeclareMathOperator{\sign}{sign}
\DeclareMathOperator*{\argmin}{arg\,min}
\DeclareMathOperator*{\argmax}{arg\,max}
\DeclareMathOperator*{\amn}{arg\,min}
\DeclareMathOperator*{\amx}{arg\,max}
\DeclareMathOperator{\cov}{Cov}
\DeclareMathOperator{\Var}{Var}
\DeclareMathOperator{\Cov}{Cov}
\DeclareMathOperator{\Corr}{Corr}
\DeclareMathOperator{\pCorr}{pCorr}
\DeclareMathOperator{\E}{\mathbb{E}}
\let\P\relax
\DeclareMathOperator{\P}{\mathbb{P}}
\renewcommand{\le}{\leqslant}
\renewcommand{\ge}{\geqslant}
\renewcommand{\leq}{\leqslant}
\renewcommand{\geq}{\geqslant}

% Распределения
\newcommand{\cN}{\mathcal{N}}
\newcommand{\cU}{\mathcal{U}}
\newcommand{\cBinom}{\mathcal{Binom}}
\newcommand{\cPois}{\mathcal{Pois}}
\newcommand{\cBeta}{\mathcal{Beta}}
\newcommand{\cGamma}{\mathcal{Gamma}}

% Множества
\def \R{\mathbb{R}}
\def \N{\mathbb{N}}
\def \Z{\mathbb{Z}}

% Другое
\newcommand{\dx}[1]{\,\mathrm{d}#1} % Для интеграла: маленький отступ и прямая d
\newcommand{\ind}[1]{\mathbbm{1}_{\{#1\}}} % Индикатор события
\newcommand{\iid}{\mathrel{\stackrel{\rm i.\,i.\,d.}\sim}}
\newcommand{\const}{\mathrm{const}}

%% Изображения
\usepackage{graphicx}
\usepackage{caption}
\usepackage{subcaption}
\usepackage{physics}
\usepackage{wrapfig} % Обтекание рисунков и таблиц текстом
\usepackage{tikz}

%% Таблицы
\usepackage{array, tabularx, tabulary, booktabs}
\usepackage{longtable}  % Длинные таблицы
\usepackage{multirow} % Слияние строк в таблице

%% Cписки
\usepackage{multicol}
\usepackage{enumitem}

%% Гиперссылки
\usepackage{xcolor}
\usepackage{hyperref}
\definecolor{linkcolor}{HTML}{8b00ff}
\hypersetup{colorlinks = true,
			linkcolor = linkcolor,
			urlcolor = linkcolor,
			citecolor = linkcolor}

%% Выравнивание
\setlength{\parskip}{0.5em} % Расстояние между абзацами
\usepackage{geometry} % Поля
\geometry{
	a4paper,
	left=20mm,
	top=20mm,
	right=20mm}
\setlength{\parindent}{0cm} % Отступ (красная строка)
\linespread{1.0} % Интерлиньяж
\usepackage[many]{tcolorbox}  

%% Оформление

\newtcolorbox{rulesbox}[1]{%
	tikznode boxed title,
	enhanced,
	arc=0mm,
	interior style={white},
	attach boxed title to top center= {yshift=-\tcboxedtitleheight/2},
	fonttitle=\bfseries,
	colbacktitle=white,coltitle=black,
	boxed title style={size=normal,colframe=white,boxrule=0pt},
	title={#1}}

% Красивый серый фон
\usepackage{framed} 
\definecolor{shadecolor}{gray}{0.9}

% Код
\newcommand{\code}[1]{{\tt #1}}

% Колонтитулы
\usepackage{fancyhdr}
\pagestyle{fancy}
\fancyhf{}
\fancyhead[L]{}
\fancyhead[R]{\thepage}

% Разделы и подразделы
\usepackage[sf, sl, outermarks]{titlesec}
\titleformat{\section}{\Large\bfseries\sffamily}{\thesection}{0.5em}{}
\titleformat{\subsection}{\large\sffamily}{\thesubsection}{0.5em}{}

% Содержание
%\usepackage{tocloft}
%\renewcommand{\cftsecfont}{\hspace{4.5em}\normalfont}
%\renewcommand{\cftsubsecfont}{\hspace{5em}\normalfont}
%\renewcommand{\cftsecpagefont}{\normalfont\hfill}
%\renewcommand{\cfttoctitlefont}{\large\normalfont\hfill}
%\renewcommand{\cftaftertoctitle}{\hfill}
%\renewcommand{\cftsecleader}{\cftdotfill{\cftdotsep}}
%\renewcommand{\cftsecafterpnum}{\hspace*{5.5em}\hfill}
%\renewcommand{\cftsubsecafterpnum}{\hspace*{5.5em}\hfill}
%\renewcommand{\cftsecaftersnum}{.}
%\renewcommand{\cftsubsecaftersnum}{.}

%% Комментарии
\usepackage{comment}

%% To-do
\usepackage{todonotes}

%% Литература
\usepackage[backend = biber,
			bibencoding = utf8, 
			sorting = nty, 
			maxcitenames = 4,
			style = numeric-verb]{biblatex}
\addbibresource{lit.bib}
\usepackage{csquotes}

%% Заголовок
\title{{\normalsize Прикладная статистика в машинном обучении} \\ \vspace{0.5em} Полезная информация о контрольной работе}
\author{\rule{15cm}{0.4pt}}




\begin{document}
	
	\maketitle

	{\Large \textbf{Список тем.}}
	
	\begin{enumerate}
		\item Метод максимального правдоподобия:
		\begin{itemize}
			\item Уметь находить ML-оценки.
			\item Знать свойства ML-оценок.
			\item Уметь строить доверительные интервалы для ML-оценок.
		\end{itemize}
		\item Тесты $LR$, $LM$, $W$.
		\item KL-дивергенция.
		\item EM-алгоритм. 
		\item Bootstrap.
		\item Геометрия МНК. 
		\item Нормальное и многомерное нормальное распределения.
	\end{enumerate}
	\vspace{2em}

	{\Large \textbf{Примеры задач.}}
	
	\textbf{NB!} Задачи, приведённые ниже, скорее, отражают уровень сложности заданий контрольной, но не являются конкретными примерами этих заданий. 
	\vspace{1em}
	
	{\large \textbf{Задача 1.}} Степан хочет проверить утверждение организаторов лотереи <<Большой-Пребольшой>>, что почти треть всех билетов являются выигрышными. Для этого он попросил $n$ своих друзей купить по 10 лотерейных билетов. Пусть $X_i$ -- число выигрышных билетов друга $i$, а $p$ -- вероятность выигрыша одного билета. 
	
	\begin{enumerate}
		\item Какое распределение имеет величина $X_i$?
		\item Запишите функцию правдоподобия для выборки $X_1$, $\ldots$, $X_n$.
		\item Найдите $\hat{p}_{ML}$.
		\item Постройте 95\%-ый доверительный интервал для $p$.
		\item Найдите оценку математического ожидания и дисперсии выигранных произвольным другом билетов. 
		\item Постройте 95\%-ый доверительный интервал для математического ожидания и дисперсии выигранных произвольным другом билетов.
		\item Дана реализация случайной выборки пяти друзей. Число выигрышных билетов оказалось равно $(3, 4, 0, 2, 6)$. Найдите значение точечной оценки вероятности выигрыша $p$. Похоже ли утверждение организаторов на правду? 
	\end{enumerate}
	\vspace{1em}
	
	{\large \textbf{Задача 2.}} Пусть регрессионная модель $y_i = \beta_0 + u_i$ оценивается методом МНК. 
	\begin{enumerate}
		\item Найдите величины $TSS$, $ESS$, $RSS$, $R^2$, scorr$(y, \hat{y})$, scov$(y, \hat{y})$.
		\item Покажите эти величины на картинке МНК.
	\end{enumerate}
	\vspace{1em}
	
	{\large \textbf{Задача 3.}} Вектор $u$ имеет стандартное нормальное распределение, $u \sim \mathcal{N}(0, I)$. Матрица $A$ такова, что $Au$ тоже имеет стандартное нормальное распределение, $Au \sim \mathcal{N}(0, I)$.
	\begin{enumerate}
		\item Выпишите уравнение, которому подчиняется матрица $A$.
		\item Чему может равняться $\det A$?
		\item Пусть $c_1$ и $c_2$ -- первый и второй столбцы матрицы $A$. Найдите $c_1'c_2$.
		\item Какое распределение имеет $||u||^2$?
	\end{enumerate}
	\vspace{1em}

	{\large \textbf{Задача 4.}} Случайная величина $X$ принимает значение $1$ с вероятностью $p$ и значение $0$ с вероятностью $(1-p)$.
	\begin{enumerate}
		\item Постройте график зависимости $H(p)$ от $p$.
		\item При каком $p$ энтропия будет максимальна? Поясните полученный результат.
	\end{enumerate}
	\vspace{1em}

	{\large \textbf{Задача 5.}} Выведите формулы для шагов EM-алгоритма для задачи разделения смеси распределений с тремя кластерами.
	\vspace{1em}
	
	{\large \textbf{Задача 6.}} Случайные величины $X$ и $Y$ независимы и имеют хи-квадрат распределение с $5$ и $10$ степенями свободы соответственно. Пусть случайная величина $Z$ равна
	\[
		Z = \dfrac{X+Y}{X}.
	\]
   Найдите такое значение $z^*$, что $\P(Z > z^*) = 0.05$. 
   \vspace{1em}
   
   {\large \textbf{Задача 7.}} Рассмотрим пространство $\R^3$ и два подпространства в нём, $W = \{(x_1, x_2, x_3) | x_1 + 2x_2 + x_3 = 0\}$ и $V = $Lin$(1, 2, 3)^T$. 
   \begin{enumerate}
   	\item Найдите $\dim V$, $\dim W$, $\dim V \cap \dim W$, $\dim V^{\perp}$, $\dim W^{\perp}$.
   	\item Найдите проекцию произвольного вектора $u$ на $V$, $W$, $V \cap W$, $V^{\perp}$, $W^{\perp}$. Найдите квадрат длины каждой проекции.
   	\item Как распределён квадрат длины проекции в каждом случае, если дополнительно известно, что вектор $u$ имеет многомерное стандартное нормальное распределение?
   \end{enumerate}
	
\end{document}