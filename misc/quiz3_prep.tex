\documentclass[10pt, a4paper]{extarticle}

%% Язык
\usepackage{cmap} % Поиск в PDF
\usepackage{mathtext} % Кириллица в формулах
\usepackage[T2A]{fontenc} % Кодировка
\usepackage[utf8]{inputenc} % Кодировка
\usepackage[english,russian]{babel} % Локализация, переносы

%% Шрифты

% Serif
%\usepackage{euscript} % Шрифт Евклид
%\usepackage{mathrsfs} % Шрифт для математики
\usepackage{libertinus}

% Sans-serif
%\renewcommand{\rmdefault}{cmss}
%\renewcommand{\ttdefault}{cmss}
%\usepackage{sfmath}

% Настройки для xelatex
%\usepackage{polyglossia} % Для выбора языка в xelatex
%\setmainlanguage{russian}
%\setotherlanguages{english}
% Ligatures=TeX is on by default
% https://tex.stackexchange.com/questions/323542/
%\setmainfont[Ligatures=TeX]{Cantarell}
%\newfontfamily{\cyrillicfonttt}{Times New Roman}
%\newfontfamily\cyrillicfont{Cantarell}[Script=Cyrillic]
%\setsansfont[Ligatures=TeX]{Cantarell}
%\newfontfamily\cyrillicfontsf{Cantarell}[Script=Cyrillic]
%\setmonofont{Courier New}
%\newfontfamily\cyrillicfonttt{Courier New}[Script=Cyrillic]

%% Математика
\usepackage{amsmath, amsfonts, amssymb, amsthm, mathtools}
\usepackage{icomma}

% Операторы
\DeclareMathOperator*\plim{plim}
\DeclareMathOperator{\sgn}{sign}
\DeclareMathOperator{\sign}{sign}
\DeclareMathOperator*{\argmin}{arg\,min}
\DeclareMathOperator*{\argmax}{arg\,max}
\DeclareMathOperator*{\amn}{arg\,min}
\DeclareMathOperator*{\amx}{arg\,max}
\DeclareMathOperator{\cov}{Cov}
\DeclareMathOperator{\Var}{Var}
\DeclareMathOperator{\Cov}{Cov}
\DeclareMathOperator{\Corr}{Corr}
\DeclareMathOperator{\pCorr}{pCorr}
\DeclareMathOperator{\E}{\mathbb{E}}
\let\P\relax
\DeclareMathOperator{\P}{\mathbb{P}}
\renewcommand{\le}{\leqslant}
\renewcommand{\ge}{\geqslant}
\renewcommand{\leq}{\leqslant}
\renewcommand{\geq}{\geqslant}

% Распределения
\newcommand{\cN}{\mathcal{N}}
\newcommand{\cU}{\mathcal{U}}
\newcommand{\cBinom}{\mathcal{Binom}}
\newcommand{\cPois}{\mathcal{Pois}}
\newcommand{\cBeta}{\mathcal{Beta}}
\newcommand{\cGamma}{\mathcal{Gamma}}

% Множества
\def \R{\mathbb{R}}
\def \N{\mathbb{N}}
\def \Z{\mathbb{Z}}

% Другое
\newcommand{\dx}[1]{\,\mathrm{d}#1} % Для интеграла: маленький отступ и прямая d
\newcommand{\ind}[1]{\mathbbm{1}_{\{#1\}}} % Индикатор события
\newcommand{\iid}{\mathrel{\stackrel{\rm i.\,i.\,d.}\sim}}
\newcommand{\const}{\mathrm{const}}

%% Изображения
\usepackage{graphicx}
\usepackage{caption}
\usepackage{subcaption}
\usepackage{physics}
\usepackage{wrapfig} % Обтекание рисунков и таблиц текстом
\usepackage{tikz}

%% Таблицы
\usepackage{array, tabularx, tabulary, booktabs}
\usepackage{longtable}  % Длинные таблицы
\usepackage{multirow} % Слияние строк в таблице

%% Cписки
\usepackage{multicol}
\usepackage{enumitem}

%% Гиперссылки
\usepackage{xcolor}
\usepackage{hyperref}
\definecolor{linkcolor}{HTML}{8b00ff}
\hypersetup{colorlinks = true,
			linkcolor = linkcolor,
			urlcolor = linkcolor,
			citecolor = linkcolor}

%% Выравнивание
\setlength{\parskip}{0.5em} % Расстояние между абзацами
\usepackage{geometry} % Поля
\geometry{
	a4paper,
	left=20mm,
	top=20mm,
	right=20mm}
\setlength{\parindent}{0cm} % Отступ (красная строка)
\linespread{1.0} % Интерлиньяж
\usepackage[many]{tcolorbox}  

%% Оформление

\newtcolorbox{rulesbox}[1]{%
	tikznode boxed title,
	enhanced,
	arc=0mm,
	interior style={white},
	attach boxed title to top center= {yshift=-\tcboxedtitleheight/2},
	fonttitle=\bfseries,
	colbacktitle=white,coltitle=black,
	boxed title style={size=normal,colframe=white,boxrule=0pt},
	title={#1}}

% Красивый серый фон
\usepackage{framed} 
\definecolor{shadecolor}{gray}{0.9}

% Код
\newcommand{\code}[1]{{\tt #1}}

% Колонтитулы
\usepackage{fancyhdr}
\pagestyle{fancy}
\fancyhf{}
\fancyhead[L]{}
\fancyhead[R]{\thepage}

% Разделы и подразделы
\usepackage[sf, sl, outermarks]{titlesec}
\titleformat{\section}{\Large\bfseries\sffamily}{\thesection}{0.5em}{}
\titleformat{\subsection}{\large\sffamily}{\thesubsection}{0.5em}{}

% Содержание
%\usepackage{tocloft}
%\renewcommand{\cftsecfont}{\hspace{4.5em}\normalfont}
%\renewcommand{\cftsubsecfont}{\hspace{5em}\normalfont}
%\renewcommand{\cftsecpagefont}{\normalfont\hfill}
%\renewcommand{\cfttoctitlefont}{\large\normalfont\hfill}
%\renewcommand{\cftaftertoctitle}{\hfill}
%\renewcommand{\cftsecleader}{\cftdotfill{\cftdotsep}}
%\renewcommand{\cftsecafterpnum}{\hspace*{5.5em}\hfill}
%\renewcommand{\cftsubsecafterpnum}{\hspace*{5.5em}\hfill}
%\renewcommand{\cftsecaftersnum}{.}
%\renewcommand{\cftsubsecaftersnum}{.}

%% Комментарии
\usepackage{comment}

%% To-do
\usepackage{todonotes}

%% Литература
\usepackage[backend = biber,
			bibencoding = utf8, 
			sorting = nty, 
			maxcitenames = 4,
			style = numeric-verb]{biblatex}
\addbibresource{lit.bib}
\usepackage{csquotes}

%% Заголовок
\title{{\normalsize Прикладная статистика в машинном обучении} \\ \vspace{0.5em} Задачи для подготовки к квизу \#3}
\author{\rule{15cm}{0.4pt}}




\begin{document}
	
	\maketitle
	
	\begin{rulesbox}{Пояснение}
		Квиз будет состоять из трёх заданий: одно на проверку гипотез (первые две задачи ниже), одно на расчёт характеристик элементов МНК (задача 3), одно на гетероскедастичность (последние две задачи). В каждое задание войдут какие-то из пунктов приведённых ниже задач с учётом времени написания в 30 минут. Обратные матрицы и прочие вещи, которые долго считать, а также критические значения для проверки гипотез, будут даны (в задачах ниже они приведены не всегда, но на квизе будут).
	\end{rulesbox}
	\vspace{1em}

	{\Large \textbf{Задача 1.}}
	
	Рассмотрим модель $y_i = \beta_0 + \beta_1X_{1i} + \beta_2X_{2i} + \beta_3X_{3i} + \beta_4X_{4i} + u_i$, которая оценивается по 3 тысячам наблюдений при помощи МНК. Оценённая модель имеет следующий вид:
	\begin{align*}
		\hat{y}_i = \underset{(1.2)}{3.04} + \underset{(0.12)}{17.1}X_{1i} - \underset{(2.7)}{8}X_{2i} + \underset{(0.24)}{0.25}X_{3i} + \underset{(0.12)}{2.99}X_{4i}.
	\end{align*}
	
	В скобках указаны стандартные ошибки оценок коэффициентов. Будем считать, что все предпосылки теоремы Гаусса-Маркова выполнены, и $u \sim \mathcal{N}(0, \sigma^2I)$.
	
	\begin{enumerate}[label=\alph*)]
		\item Проверьте каждый коэффициент на значимость на 5\% уровне.
		\item Проверьте гипотезу
		\[
		\begin{cases}
			H_0: \beta_1 = 15, \\
			H_1: \beta_1 \ne 15
		\end{cases}
		\]
		на уровне значимости 5\%.
		\item Проверьте гипотезу
		\[
		\begin{cases}
			H_0: \beta_2 = -10, \\
			H_1: \beta_2 > -10
		\end{cases}
		\]
		на уровне значимости 5\%.
		\item Постройте 95\%-ый доверительный интервал для $\beta_4$. 
		\item Найдите $\hat{y}_{3001} | X_1 = X_2 = X_4 = 0, X_3 = 4$. 
		\item Постройте 95\%-ый доверительный интервал для $\E(y_{3001} | X_1 = X_2 = X_4 = 0, X_3 = 4)$, если 
		\[
		\hat{\Var}(\hat{\beta}) = \begin{pmatrix}
			  1.44 & -0.04 & 0.001 & -0.001  & 0.47 \\
			-0.04  & 0.0144 & 0.09 & 1 & 0.9 \\
			0.001  & 0.09 & 7.29 & 0.017 & -0.48 \\
			-0.001 & 1 & 0.017 & 0.0576 & -0.002 \\
			0.47 & 0.9 & -0.48 & -0.002 & 0.0144
		\end{pmatrix}
		\]
	\end{enumerate}
	\vspace{1em}
	\newpage

	{\Large \textbf{Задача 2.}}
	
	Рассмотрим модель $y_i = \beta_0 + \beta_1x_{1i} + \beta_2x_{2i} + u_i$, оцениваемую при помощи МНК. Предположим, что все предпосылки теоремы Гаусса-Маркова выполнены, и $u \sim \mathcal{N}(0, \sigma^2I)$. Модель оценили на следующих данных:
	\[
	X = \begin{bmatrix}
		1 & 2 & 30 \\
		1 & 5 & 20 \\
		1 & 2 & 20 \\
		1 & 5 & 30 \\
	\end{bmatrix},
	y = \begin{bmatrix}
		10 \\
		5 \\
		10 \\
		1
	\end{bmatrix}.
	\]
	
	Оказалось, что $\hat{\beta} = \begin{pmatrix}
		9.25 & -1.5 & 0.05
	\end{pmatrix}'$.

	\begin{enumerate}[label=\alph*)]
		\item Проверьте регрессию на значимость в целом на уровне значимости 5\%.
		\item Проверьте гипотезу
		\[
		\begin{cases}
			H_0: \beta_1 = \beta_2, \\
			H_1: \beta_1 \ne \beta_2
		\end{cases}
		\]
		на уровне значимости 5\%.
	\end{enumerate}
	\vspace{1em}
	
	{\Large \textbf{Задача 3.}}
	
	Рассмотрим линейную модель $y = X\beta + u$, оцениваемую при помощи МНК. Пусть $\E(u) = 0$, $\Var(u) = \sigma^2 I$, число наблюдений равно $n$, число регрессоров, включая константный, равно $k$. 
	
	На семинаре мы начали заполнять матрицу характеристик элементов МНК:
	\begin{center}
		\begin{tabular}{c|ccccc}
			$\Var(\cdot)$ & $y$ & $\hat{y}$ & $\hat{\beta}$ & $\hat{u}$ & $u$ \\
			\hline
			$y$ & $\underset{n \times n}{\sigma^2I}$ &&&& \\
			$\hat{y}$ &&$\ldots$&&& \\
			$\hat{\beta}$ &&&$\underset{k\times k}{(X'X)^{-1}\sigma^2}$& $\underset{k \times n}{0}$& \\
			$\hat{u}$ &&&&$\ldots$& \\
			$u$ &&&&&$\ldots$ \\
			\hline
			$\E(\cdot)$ & $\underset{n\times1}{X\beta}$&&$\underset{k \times 1}{\beta}$&& \\
			\hline
		\end{tabular}
	\end{center}
	
	Закончите заполнение матрицы. На квизе может быть любой элемент этой матрицы. Для каждого элемента укажите размеры.
	\vspace{1em}
	
	{\Large \textbf{Задача 4.}}
	
	Рассмотрим модель $y = X\beta + u$, оцениваемую при помощи МНК по 150 наблюдениям. Известно, что
	\[
	X = \begin{bmatrix}
	1 & 4 & 1 \\
	1 & 9 & 3 \\
	1 & 1 & 1
	\end{bmatrix},
	X'X = \begin{bmatrix}
		3 & 14 & 5 \\
		14 & 98 & 32 \\
		5 & 32 & 11
	\end{bmatrix},
	(X'X)^{-1} = \begin{bmatrix}
		1.5 & 0.17 & -1.17 \\
		0.17 & 0.22 & -0.72 \\
		-1.16 & -0.72 & 2.72
	\end{bmatrix},
	y = \begin{bmatrix}
		1 \\
		2 \\
		3 \\
	\end{bmatrix}
	\]
	
	Также известно, что $\Var(u) = \sigma^2A$, где $A = \mathrm{diag}(3, 4, 5)$. Все остальные предпосылки теоремы Гаусса-Маркова выполнены.
	
	\textit{\textbf{Замечание:}} можно оставить выражения в матрицах, цель задания -- разобраться где какая формула, как она получается и что куда подставлять.
	
	\begin{enumerate}[label = \alph*)]
		\item Скорректируйте гетероскедастичность в модели и выпишите формулу для оценок скорректированной модели.
		\item Выведите выражение для оценки ковариационной матрицы оценок МНК и найдите её в числах.
		\item Выведите выражение для оценки ковариационной матрицы оценок в скорректированной модели и найдите её в числах.
	\end{enumerate}
	\vspace{1em}

	{\Large \textbf{Задача 5.}}
	
	Рассмотрим модель парной регрессии
	\[
	y_i = \beta_0 + \beta_1x_i + u_i,
	\]
	оцениваемую при помощи МНК. Пусть известно, что 
	
	\begin{center}
	\begin{tabular}{c|c}
		$y_i$ & $x_i$ \\
		\hline
		1 & 1 \\
		2 & 1 \\
		1 & 4 \\
		3 & 2
	\end{tabular}
	\end{center}

	\[
	(X'X)^{-1} = \begin{bmatrix}
		0.92 & -0.33 \\
		-0.33 & 0.17
	\end{bmatrix}
	\]

	\begin{enumerate}[label = \alph*)]
		\item Найдите $\hat{\beta}_{OLS}$.		
		\item Постройте 95\%-ый доверительный интервал для $\hat{\beta}_0$, используя стандартные ошибки $HC_0$.
		\item Постройте 95\%-ый доверительный интервал для $\hat{\beta}_1$, используя стандартные ошибки $HC_3$.
	\end{enumerate}
	
\end{document}