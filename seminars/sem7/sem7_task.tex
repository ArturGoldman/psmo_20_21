\documentclass[10pt, a4paper]{extarticle}

%% Язык
\usepackage{cmap} % Поиск в PDF
\usepackage{mathtext} % Кириллица в формулах
\usepackage[T2A]{fontenc} % Кодировка
\usepackage[utf8]{inputenc} % Кодировка
\usepackage[english,russian]{babel} % Локализация, переносы

%% Шрифты

% Serif
%\usepackage{euscript} % Шрифт Евклид
%\usepackage{mathrsfs} % Шрифт для математики
\usepackage{libertinus}

% Sans-serif
%\renewcommand{\rmdefault}{cmss}
%\renewcommand{\ttdefault}{cmss}
%\usepackage{sfmath}

% Настройки для xelatex
%\usepackage{polyglossia} % Для выбора языка в xelatex
%\setmainlanguage{russian}
%\setotherlanguages{english}
% Ligatures=TeX is on by default
% https://tex.stackexchange.com/questions/323542/
%\setmainfont[Ligatures=TeX]{Cantarell}
%\newfontfamily{\cyrillicfonttt}{Times New Roman}
%\newfontfamily\cyrillicfont{Cantarell}[Script=Cyrillic]
%\setsansfont[Ligatures=TeX]{Cantarell}
%\newfontfamily\cyrillicfontsf{Cantarell}[Script=Cyrillic]
%\setmonofont{Courier New}
%\newfontfamily\cyrillicfonttt{Courier New}[Script=Cyrillic]

%% Математика
\usepackage{amsmath, amsfonts, amssymb, amsthm, mathtools}
\usepackage{icomma}

\newtheorem{theor}{Теорема}
\newtheorem{defn}{Определение}

% Операторы
\DeclareMathOperator*\plim{plim}
\DeclareMathOperator{\sgn}{sign}
\DeclareMathOperator{\sign}{sign}
\DeclareMathOperator*{\argmin}{arg\,min}
\DeclareMathOperator*{\argmax}{arg\,max}
\DeclareMathOperator*{\amn}{arg\,min}
\DeclareMathOperator*{\amx}{arg\,max}
\DeclareMathOperator{\cov}{Cov}
\DeclareMathOperator{\Var}{Var}
\DeclareMathOperator{\Cov}{Cov}
\DeclareMathOperator{\Corr}{Corr}
\DeclareMathOperator{\pCorr}{pCorr}
\DeclareMathOperator{\E}{\mathbb{E}}
\let\P\relax
\DeclareMathOperator{\P}{\mathbb{P}}
\renewcommand{\le}{\leqslant}
\renewcommand{\ge}{\geqslant}
\renewcommand{\leq}{\leqslant}
\renewcommand{\geq}{\geqslant}

% Распределения
\newcommand{\cN}{\mathcal{N}}
\newcommand{\cU}{\mathcal{U}}
\newcommand{\cBinom}{\mathcal{Binom}}
\newcommand{\cPois}{\mathcal{Pois}}
\newcommand{\cBeta}{\mathcal{Beta}}
\newcommand{\cGamma}{\mathcal{Gamma}}

% Множества
\def \R{\mathbb{R}}
\def \N{\mathbb{N}}
\def \Z{\mathbb{Z}}

% Другое
\newcommand{\dx}[1]{\,\mathrm{d}#1} % Для интеграла: маленький отступ и прямая d
\newcommand{\ind}[1]{\mathbbm{1}_{\{#1\}}} % Индикатор события
\newcommand{\iid}{\mathrel{\stackrel{\rm i.\,i.\,d.}\sim}}
\newcommand{\const}{\mathrm{const}}

%% Изображения
\usepackage{graphicx}
\usepackage{caption}
\usepackage{subcaption}
\usepackage{physics}
\usepackage{wrapfig} % Обтекание рисунков и таблиц текстом
\usepackage{tikz}

%% Таблицы
\usepackage{array, tabularx, tabulary, booktabs}
\usepackage{longtable}  % Длинные таблицы
\usepackage{multirow} % Слияние строк в таблице

%% Cписки
\usepackage{multicol}
\usepackage{enumitem}

%% Гиперссылки
\usepackage{xcolor}
\usepackage{hyperref}
\definecolor{linkcolor}{HTML}{8b00ff}
\hypersetup{colorlinks = true,
			linkcolor = linkcolor,
			urlcolor = linkcolor,
			citecolor = linkcolor}

%% Выравнивание
\setlength{\parskip}{0.5em} % Расстояние между абзацами
\usepackage{geometry} % Поля
\geometry{
	a4paper,
	left=20mm,
	top=20mm,
	right=20mm}
\setlength{\parindent}{0cm} % Отступ (красная строка)
\linespread{1.0} % Интерлиньяж
\usepackage[many]{tcolorbox}  

%% Оформление

\newtcolorbox{rulesbox}[1]{%
	tikznode boxed title,
	enhanced,
	arc=0mm,
	interior style={white},
	attach boxed title to top center= {yshift=-\tcboxedtitleheight/2},
	fonttitle=\bfseries,
	colbacktitle=white,coltitle=black,
	boxed title style={size=normal,colframe=white,boxrule=0pt},
	title={#1}}

% Красивый серый фон
\usepackage{framed} 
\definecolor{shadecolor}{gray}{0.9}

% Код
\newcommand{\code}[1]{{\tt #1}}

% Колонтитулы
\usepackage{fancyhdr}
\pagestyle{fancy}
\fancyhf{}
\fancyhead[L]{}
\fancyhead[R]{\thepage}

% Разделы и подразделы
\usepackage[sf, sl, outermarks]{titlesec}
\titleformat{\section}{\Large\bfseries\sffamily}{\thesection}{0.5em}{}
\titleformat{\subsection}{\large\sffamily}{\thesubsection}{0.5em}{}

% Содержание
%\usepackage{tocloft}
%\renewcommand{\cftsecfont}{\hspace{4.5em}\normalfont}
%\renewcommand{\cftsubsecfont}{\hspace{5em}\normalfont}
%\renewcommand{\cftsecpagefont}{\normalfont\hfill}
%\renewcommand{\cfttoctitlefont}{\large\normalfont\hfill}
%\renewcommand{\cftaftertoctitle}{\hfill}
%\renewcommand{\cftsecleader}{\cftdotfill{\cftdotsep}}
%\renewcommand{\cftsecafterpnum}{\hspace*{5.5em}\hfill}
%\renewcommand{\cftsubsecafterpnum}{\hspace*{5.5em}\hfill}
%\renewcommand{\cftsecaftersnum}{.}
%\renewcommand{\cftsubsecaftersnum}{.}

%% Комментарии
\usepackage{comment}

%% To-do
\usepackage{todonotes}

%% Литература
\usepackage[backend = biber,
			bibencoding = utf8, 
			sorting = nty, 
			maxcitenames = 4,
			style = numeric-verb]{biblatex}
\addbibresource{lit.bib}
\usepackage{csquotes}

%% Заголовок
\title{{\normalsize Прикладная статистика в машинном обучении} \\ \vspace{0.5em} Семинар 7}
\author{\rule{15cm}{0.4pt}}




\begin{document}
	
	\maketitle

	{\Large \textbf{Задача 1.} Теорема Хершелла-Максвелла.} 
	
	Рассмотрим замкнутую плоскую фигуру внутри которой случайно летают частицы. Обозначим $V = (V_x, V_y)'$ -- вектор скоростей случайно выбранной частицы.
	
	Предположим, что
	\begin{enumerate}
		\item Распределение вектора $V$ не должно меняться при его повороте на любой угол (то есть не зависит от направления вектора).
		\item $V_x$ и $V_y$ независимы.
		\item $\Var(V_x) = \Var(V_y) = 1$.
		\item $f(v_x, v_y)$ существует и непрерывна.
	\end{enumerate}
	
	\begin{enumerate}[label=\textbf{\alph*)}]
		\item Найдите координаты вектора $V'$, который получается поворотом вектора $V$ на $90^{\circ}$ против часовой стрелки.
		\item Докажите, что распределения $V_x$, $V_y$ и $-V_y$ совпадают.
		\item По предположению 3 понятно, что $\Var(V_x) = \Var(V_y) = 1$. Покажите, что $\E(V_x) = \E(V_y) = 0$.
		\item Докажите, что совместная функция плотности $f_V(v_x, v_y)$ представима в виде
		\[
		f_V(v_x, v_y) = h(v_x^2 + v_y^2),
		\]
		где $h$ -- некоторая функция. Сделайте вывод из этого утверждения.
		\item Покажите, что совместная функция плотности $f_V(v_x, v_y)$ представима в виде
		\[
		f_V(v_x, v_y) = g(v_x^2)g(v_y^2),
		\]
		где $g$ -- некоторая функция.
		\item Докажите, что выражение $h'(t)/h(t)$ равно константе.
		\item Найдите функцию $h(t)$ с точностью до константы.
		\item Выпишите $f_V(v_x, v_y)$ с точностью до константы.
	\end{enumerate}
	\vspace{1em}
	
	{\Large \textbf{Задача 2.} Независимость длин проекций.}
	
	Пусть вектор $u \in \R^3$ имеет многомерное стандартное нормальное распределение.
	\begin{enumerate}[label=\textbf{\alph*)}]
		\item Найдите проекцию вектора $u$ на подпространство $V = \{(x_1, x_2, x_3) \text{ }| \text{ } x_1 = x_2 = x_3\}$.
		\item Найдите проекцию вектора $u$ на подпространство
		$W = \{(x_1, x_2, x_3) \text{ }| \text{ } x_2 = 0, x_1 = -x_3\}$.
		\item Являются ли $V$ и $W$ ортогональными?
		\item Убедитесь, что $\Vert\hat{u}_V\Vert$ и $\Vert\hat{u}_W\Vert$ независимы.
	\end{enumerate} 
	\vspace{1em}
	
	{\Large \textbf{Задача 3.} Распределение хи-квадрат.}
	\begin{theor}
		Пусть вектор $u \in \R^n$ имеет многомерное стандартное нормальное распределение. Пусть $V \subset \R^n$, $\dim V = k$. Тогда $\Vert\hat{u}_V\Vert^2 \sim \chi^2_k$.
	\end{theor}

	\begin{enumerate}[label=\textbf{\alph*)}]
		\item Пусть вектор $u \in \R^3$ имеет многомерное стандартное нормальное распределение. Найдите проекцию и распределение квадрата длины проекции вектора $u$ на подпространство $V = \{(x_1, x_2, x_3 \text{ } | \text{ } x_3 = 2x_1 + x_2)\}$.
		\item Найдите распределение квадрата длины проекции вектора $u$ на подпространство $V^{\perp}$.
	\end{enumerate}
	\vspace{1em}

	{\Large \textbf{Задача 4.} F-распределение.}
	\begin{defn}
		Пусть случайные величины $X \sim \chi^2_{a}$, $Y \sim \chi^2_{b}$, и $X$ и $Y$ независимы. Тогда случайная величина $Z$ имеет распределение Фишера с $a$ и $b$ степенями свободы:
		\[
		Z = \dfrac{X/a}{Y/b} \sim F_{a, b}.
		\]
	\end{defn}
	
	Пусть вектор $u \in \R^n$ имеет многомерное стандартное нормальное распределение. Пусть $V$ и $W$ -- ортогональные подпространства в $\R^n$, $\dim V = k$. $\dim W = m$.
	\begin{enumerate}[label=\textbf{\alph*)}]
		\item Постройте проекции вектора $u$ на $V$ и $W$. Как распределены квадраты длин этих проекций?
		\item Постройте проекцию вектора $u$ на $(V + W)$.
		\item Покажите на рисунке угол, квадрат тангенса которого равен отношению квадратов длин проекций $u$ на $V$ и $W$. Обозначим этот угол как $\alpha$.
		\item Какое распределение имеет величина
		\[
		\dfrac{m}{k}\tan^2 \alpha ?
		\]
		\item Поясните идею сравнения прогнозов моделей при помощи $F$-распределения.
	\end{enumerate}
	\vspace{1em}

	{\Large \textbf{Задача 5.} Применение F-распределения.}
	\begin{theor}
		Пусть имеются UR-модель и R-модель и тестируется гипотеза 
		\[
		H_0: \text{Верны и UR- и R-модели}
		\]
		против
		\[
		H_A: \text{UR-модель верна, а R-модель неверна.}
		\]
		Тогда $H_0$ отвергается на уровне значимости $\alpha$, если $F$-статистика
		\[
		F = \dfrac{(RSS_{R} - RSS_{UR})/(\text{df}_{R} - \text{df}_{UR})}{RSS_{UR} / \text{df}_{UR}}
		\]
		превышает критическое значение $F_{\alpha}$. Здесь $df$ -- число степеней свободы в соответствующей модели, $df = n - k$, где $n$ -- число наблюдений, $k$ -- число регрессоров. 
	\end{theor}
	
	Пусть UR-модель задаётся следующим образом:
	\[
	y_i = \beta_0 + \beta_1x_i + \beta_2z_i + u_i.
	\]
	Тестируется следующая R-модель:
	\[
	y_i = \beta_0 + \beta_1(x_i + z_i) + u_i.
	\]
	
	Предположим, что $u_i \sim \mathcal{N}(0, 1)$ и независимы.
	
	\begin{enumerate}[label=\textbf{\alph*)}]
		\item Постройте подпространства $V_{UR}$ и $V_{R}$ и проекции вектора $y$ на них.
		\item Покажите $RSS_{UR}$ и $RSS_{R}$. Обозначим угол между ними как $\alpha$.
		\item Покажите, что $RSS_{R} - RSS_{UR} = \Vert \hat{y}_{UR} - \hat{y}_{R} \Vert^2$.
		\item Выразите $\tan^2 \alpha$ через $y$,  $\hat{y}_{UR}$, $\hat{y}_{R}$.
		\item Рассмотрим векторы $y - \hat{y}_{UR}$ и $\hat{y}_{UR} - \hat{y}_{R}$. Найдите размерности подпространств, в которых они лежат. 
		\item Выпишите $F$-статистику в геометрическом и в классическом смыслах.
	\end{enumerate}
	
\end{document}